% ============================================
% CHAPITRE 6 : CONCLUSION
% ============================================

\section{Synthèse des résultats}

Ce projet avait pour objectif de développer un système de détection de fraudes sur les transactions par chèque, avec deux critères d'optimisation différents.

\subsection{Partie 1 : Optimisation du F1-Score}

\begin{table}[H]
\centering
\caption{Résumé Partie 1}
\begin{tabular}{ll}
\toprule
\textbf{Élément} & \textbf{Résultat} \\
\midrule
Meilleur modèle & LightGBM + ADASYN \\
F1-Score (seuil par défaut) & 0.0685 \\
F1-Score (seuil optimisé = 0.74) & 0.1070 \\
Fraudes détectées & 583 / 6 573 (8.9\%) \\
Amélioration par optimisation du seuil & +56\% \\
\bottomrule
\end{tabular}
\end{table}

\subsection{Partie 2 : Optimisation de la Marge}

\begin{table}[H]
\centering
\caption{Résumé Partie 2}
\begin{tabular}{ll}
\toprule
\textbf{Élément} & \textbf{Résultat} \\
\midrule
Meilleur modèle & XGBoost optimisé \\
Marge obtenue & 2 010 033 € \\
Marge de référence (sans détection) & 1 941 852 € \\
Amélioration & +68 181 € (+3.5\%) \\
Fraudes détectées & 628 / 6 573 (9.6\%) \\
\bottomrule
\end{tabular}
\end{table}

\section{Enseignements clés}

\subsection{Sur le déséquilibre des classes}

Le fort déséquilibre des classes (1:166) constitue le défi majeur de ce projet. Les techniques de rééchantillonnage (SMOTE, ADASYN) et l'optimisation du seuil de décision sont essentielles pour obtenir des performances acceptables.

\subsection{Sur l'optimisation du seuil}

L'optimisation du seuil de décision s'est révélée \textbf{cruciale} dans les deux parties :
\begin{itemize}
    \item Partie 1 : +56\% de F1-Score
    \item Partie 2 : Différence entre gain et perte par rapport à la référence
\end{itemize}

Le seuil par défaut de 0.5 n'est \textbf{jamais optimal} pour les données déséquilibrées.

\subsection{Sur les objectifs différents}

Les deux parties illustrent l'importance de définir clairement l'objectif métier :
\begin{itemize}
    \item Le F1-Score traite toutes les erreurs de manière égale
    \item La marge prend en compte les coûts réels, qui varient selon le montant
\end{itemize}

Le meilleur modèle pour le F1-Score (LightGBM + ADASYN) n'est pas le meilleur pour la marge (XGBoost optimisé).

\section{Limites du projet}

\subsection{Taux de détection}

Malgré les optimisations, le taux de détection reste faible :
\begin{itemize}
    \item Partie 1 : 8.9\% des fraudes détectées
    \item Partie 2 : 9.6\% des fraudes détectées
\end{itemize}

Cela s'explique par les corrélations faibles entre les features et la variable cible (maximum ~0.15).

\subsection{Drift temporel}

Le taux de fraude est plus élevé dans l'ensemble de test (0.88\%) que dans l'ensemble d'entraînement (0.60\%), suggérant une évolution des patterns de fraude dans le temps. Un système en production nécessiterait un réentraînement régulier.

\subsection{Données limitées}

Certaines informations potentiellement utiles ne sont pas disponibles :
\begin{itemize}
    \item Historique du client
    \item Informations géographiques
    \item Comportement temporel (heure, jour de la semaine)
\end{itemize}

\section{Perspectives}

\subsection{Améliorations techniques}

\begin{enumerate}
    \item \textbf{Feature engineering} : Créer de nouvelles variables (ratios, agrégations temporelles)
    \item \textbf{Modèles plus complexes} : Réseaux de neurones, autoencoders pour la détection d'anomalies
    \item \textbf{Optimisation bayésienne} : Pour le tuning des hyperparamètres
    \item \textbf{Validation croisée temporelle} : Pour une évaluation plus robuste
\end{enumerate}

\subsection{Déploiement}

\begin{enumerate}
    \item \textbf{Système temps réel} : Intégration dans le processus de validation des chèques
    \item \textbf{Monitoring} : Surveillance des performances et détection du drift
    \item \textbf{Réentraînement automatique} : Mise à jour périodique du modèle
    \item \textbf{Interface utilisateur} : Dashboard pour les analystes fraude
\end{enumerate}

\section{Conclusion finale}

Ce projet démontre la faisabilité d'un système de détection de fraudes par chèque basé sur le Machine Learning. Le système développé permet d'améliorer la marge du commerçant de \textbf{+3.5\%} (+68 181 €) par rapport à une stratégie sans détection.

Les principales leçons tirées sont :
\begin{enumerate}
    \item L'importance des techniques de gestion du déséquilibre des classes
    \item Le rôle crucial de l'optimisation du seuil de décision
    \item La nécessité d'aligner le critère d'optimisation avec l'objectif métier
\end{enumerate}

Bien que les performances de détection restent modestes (environ 10\% des fraudes détectées), le gain économique justifie la mise en place d'un tel système.
