% ============================================
% RÉSUMÉ
% ============================================

Ce projet s'inscrit dans le cadre du Master 2 SISE et porte sur la détection de fraudes dans les transactions par chèque. Les données proviennent du Fichier National des Chèques Irréguliers (FNCI), géré par la Banque de France.

\vspace{0.5cm}

\textbf{Objectifs :}
\begin{itemize}
    \item \textbf{Partie 1} : Maximiser le F-mesure (F1-Score) pour détecter le maximum de fraudes
    \item \textbf{Partie 2} : Maximiser la marge du commerçant en utilisant une matrice de coûts asymétrique
\end{itemize}

\vspace{0.5cm}

\textbf{Données :}
\begin{itemize}
    \item 4 646 773 transactions (23 variables)
    \item Split temporel : entraînement (février-août 2017) / test (septembre-novembre 2017)
    \item Fort déséquilibre : 0.60\% de fraudes dans l'ensemble d'entraînement
\end{itemize}

\vspace{0.5cm}

\textbf{Résultats principaux :}

\begin{table}[H]
\centering
\begin{tabular}{lcc}
\toprule
\textbf{Critère} & \textbf{Meilleur modèle} & \textbf{Performance} \\
\midrule
F1-Score & LightGBM + ADASYN & 0.107 (seuil optimisé) \\
Marge & XGBoost optimisé & +68 181 € vs référence \\
\bottomrule
\end{tabular}
\end{table}

\vspace{0.5cm}

\textbf{Conclusion :} Le système de détection de fraude permet d'améliorer la marge du commerçant de +3.5\% par rapport à une stratégie sans détection.

\vspace{0.5cm}

\textbf{Mots-clés :} Détection de fraude, Classification déséquilibrée, Machine Learning, SMOTE, ADASYN, Random Forest, XGBoost, LightGBM, Matrice de coûts
