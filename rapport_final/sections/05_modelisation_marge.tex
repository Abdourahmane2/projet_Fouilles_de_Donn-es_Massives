% ============================================
% CHAPITRE 5 : MODÉLISATION - PARTIE 2 (MARGE)
% ============================================

\section{Objectif}

L'objectif de cette seconde partie est de maximiser la \textbf{marge du commerçant} en tenant compte des coûts réels associés à chaque type de décision.

Contrairement à la Partie 1 où toutes les erreurs étaient traitées de manière égale, nous utilisons ici une \textbf{matrice de coûts asymétrique} qui reflète la réalité économique.

\section{Matrice de coûts}

Pour une transaction de montant $m$ :

\begin{table}[H]
\centering
\caption{Matrice de coûts}
\begin{tabular}{lll}
\toprule
\textbf{Situation} & \textbf{Description} & \textbf{Gain/Perte} \\
\midrule
TN (Vrai Négatif) & Client normal accepté & $+5\% \times m$ \\
FP (Faux Positif) & Client normal refusé & $-70\% \times 5\% \times m$ \\
TP (Vrai Positif) & Fraude détectée & $0$ \\
FN (Faux Négatif) & Fraude manquée & Variable selon $m$ \\
\bottomrule
\end{tabular}
\end{table}

\subsection{Détail des coûts FN (Faux Négatifs)}

La perte associée à une fraude non détectée dépend du montant de la transaction :

\begin{table}[H]
\centering
\caption{Perte FN selon le montant}
\begin{tabular}{lc}
\toprule
\textbf{Tranche de montant} & \textbf{Perte} \\
\midrule
$m \leq 20$ € & 0\% \\
$20 < m \leq 50$ € & 20\% de $m$ \\
$50 < m \leq 100$ € & 30\% de $m$ \\
$100 < m \leq 200$ € & 50\% de $m$ \\
$m > 200$ € & 80\% de $m$ \\
\bottomrule
\end{tabular}
\end{table}

\textbf{Interprétation :} Les fraudes sur les petits montants ($\leq$ 20 €) sont couvertes par l'assurance, donc sans perte. Plus le montant augmente, plus la perte est importante.

\section{Calcul de la marge}

La marge totale est calculée comme la somme des gains/pertes sur toutes les transactions :

\begin{equation}
    Marge_{totale} = \sum_{i=1}^{n} gain_i(y_i, \hat{y}_i, m_i)
\end{equation}

où $gain_i$ est déterminé par la matrice de coûts en fonction de :
\begin{itemize}
    \item $y_i$ : Vraie étiquette (0 = normal, 1 = fraude)
    \item $\hat{y}_i$ : Prédiction du modèle
    \item $m_i$ : Montant de la transaction
\end{itemize}

\section{Référence : Sans détection de fraude}

Avant d'évaluer les modèles, nous calculons la marge de référence si le commerçant acceptait \textbf{toutes} les transactions sans système de détection :

\begin{table}[H]
\centering
\caption{Marge de référence (tout accepter)}
\begin{tabular}{ll}
\toprule
\textbf{Composante} & \textbf{Montant} \\
\midrule
Gain TN (clients normaux) & +2 294 459 € \\
Perte FP (fausses alertes) & 0 € \\
Perte FN (fraudes manquées) & -352 608 € \\
\midrule
\textbf{Marge totale} & \textbf{1 941 852 €} \\
\bottomrule
\end{tabular}
\end{table}

Cette valeur sert de \textbf{point de comparaison} : un système de détection n'est utile que s'il améliore cette marge.

\section{Méthodes testées}

Nous avons testé plusieurs approches :

\begin{enumerate}
    \item \textbf{LightGBM + ADASYN} (meilleur modèle de la Partie 1)
    \item \textbf{LightGBM + ADASYN avec seuil optimisé} pour la marge
    \item \textbf{XGBoost avec différents poids de classe}
    \item \textbf{Random Forest + SMOTE avec seuil optimisé}
    \item \textbf{Seuil adaptatif} (seuils différents selon le montant)
    \item \textbf{Ensemble} (moyenne des probabilités de plusieurs modèles)
\end{enumerate}

\section{Optimisation du seuil pour la marge}

Contrairement à la Partie 1 où on maximisait le F1-Score, ici on cherche le seuil qui maximise la marge en euros.

\textbf{Méthode :} Pour chaque seuil $s \in [0.01, 0.99]$, calculer la marge totale et retenir le seuil donnant la meilleure marge.

\section{Stratégie de seuil adaptatif}

L'idée est d'utiliser des seuils différents selon le montant de la transaction :
\begin{itemize}
    \item Pour les petits montants : seuil élevé (moins prudent)
    \item Pour les gros montants : seuil bas (plus prudent)
\end{itemize}

\textbf{Justification :} Le coût d'une fraude manquée est beaucoup plus élevé pour les gros montants, donc il vaut mieux être plus prudent.

\section{Résultats}

\begin{table}[H]
\centering
\caption{Comparaison des marges par méthode}
\begin{tabular}{lrrrr}
\toprule
\textbf{Méthode} & \textbf{Marge (€)} & \textbf{Gain TN} & \textbf{Perte FP} & \textbf{Perte FN} \\
\midrule
XGBoost optimisé & \textbf{2 010 033} & 2 269 681 & -17 345 & -242 304 \\
Ensemble & 1 998 133 & 2 277 594 & -11 806 & -267 656 \\
LightGBM (seuil opt) & 1 988 509 & 2 273 072 & -14 971 & -269 591 \\
RF + SMOTE optimisé & 1 978 646 & 2 289 693 & -3 337 & -307 710 \\
\textbf{Référence} & \textbf{1 941 852} & 2 294 459 & 0 & -352 608 \\
Seuil adaptatif & 1 665 461 & 2 000 923 & -205 476 & -129 986 \\
LightGBM (seuil 0.5) & 1 641 927 & 1 994 041 & -210 293 & -141 821 \\
\bottomrule
\end{tabular}
\end{table}

\section{Analyse des résultats}

\subsection{Meilleur modèle : XGBoost optimisé}

\begin{table}[H]
\centering
\caption{Performance du meilleur modèle}
\begin{tabular}{ll}
\toprule
\textbf{Métrique} & \textbf{Valeur} \\
\midrule
Marge totale & 2 010 033 € \\
Amélioration vs référence & +68 181 € (+3.5\%) \\
Fraudes détectées (TP) & 628 sur 6 573 (9.6\%) \\
Fraudes manquées (FN) & 5 945 \\
Fausses alertes (FP) & Perte de 17 345 € \\
\bottomrule
\end{tabular}
\end{table}

\subsection{Observations clés}

\begin{enumerate}
    \item \textbf{Le système de détection améliore la marge} : +68 181 € par rapport à la stratégie "tout accepter".
    
    \item \textbf{Le seuil par défaut (0.5) fait perdre de l'argent} : LightGBM avec seuil 0.5 donne une marge de 1 641 927 €, inférieure à la référence (1 941 852 €).
    
    \item \textbf{L'optimisation du seuil est cruciale} : Le même modèle avec un seuil optimisé passe de 1 641 927 € à 1 988 509 €.
    
    \item \textbf{Le seuil adaptatif n'est pas optimal} : Bien qu'il détecte plus de fraudes (1 349 TP), il génère trop de fausses alertes (-205 476 €).
    
    \item \textbf{L'équilibre est clé} : Le meilleur modèle (XGBoost) trouve le bon compromis entre :
    \begin{itemize}
        \item Réduire les pertes FN : de 352 608 € à 242 304 € (-110 304 €)
        \item Limiter les pertes FP : seulement 17 345 € de manque à gagner
    \end{itemize}
\end{enumerate}

\subsection{Comparaison avec la Partie 1}

\begin{table}[H]
\centering
\caption{Comparaison des objectifs}
\begin{tabular}{lcc}
\toprule
\textbf{Critère} & \textbf{Partie 1 (F1)} & \textbf{Partie 2 (Marge)} \\
\midrule
Meilleur modèle & LightGBM + ADASYN & XGBoost optimisé \\
Seuil optimal & 0.74 & Variable \\
Fraudes détectées & 583 (8.9\%) & 628 (9.6\%) \\
Objectif atteint & F1 = 0.107 & +68 181 € \\
\bottomrule
\end{tabular}
\end{table}

\section{Conclusion de la Partie 2}

\begin{itemize}
    \item Le système de détection de fraude \textbf{améliore la marge} du commerçant de \textbf{+68 181 €} (+3.5\%).
    
    \item Le meilleur modèle est \textbf{XGBoost avec optimisation du seuil}.
    
    \item L'optimisation du seuil est \textbf{encore plus importante} que dans la Partie 1 : un mauvais seuil peut faire perdre de l'argent par rapport à l'absence de détection.
    
    \item La stratégie de seuil adaptatif, bien qu'intuitive, n'est pas optimale dans notre cas car elle génère trop de fausses alertes.
    
    \item Le gain de 68 181 € justifie économiquement la mise en place d'un système de détection de fraude.
\end{itemize}
