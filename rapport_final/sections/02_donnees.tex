% ============================================
% CHAPITRE 2 : DESCRIPTION DES DONNÉES
% ============================================

\section{Source des données}

Les données utilisées dans ce projet proviennent du \textbf{Fichier National des Chèques Irréguliers (FNCI)}, géré par la Banque de France. Ce fichier recense les informations relatives aux transactions par chèque effectuées en France.

\section{Vue d'ensemble}

\begin{table}[H]
\centering
\caption{Caractéristiques générales du jeu de données}
\begin{tabular}{ll}
\toprule
\textbf{Caractéristique} & \textbf{Valeur} \\
\midrule
Nombre total de transactions & 4 646 773 \\
Nombre de variables & 23 \\
Variable cible & FlagImpaye (0 = normal, 1 = fraude) \\
Période couverte & Février 2017 - Novembre 2017 \\
\bottomrule
\end{tabular}
\end{table}

\section{Description des variables}

Le jeu de données contient 23 variables réparties en plusieurs catégories :

\subsection{Variable cible}

\begin{itemize}
    \item \textbf{FlagImpaye} : Indicateur de fraude (1 = impayé/fraude, 0 = normal)
\end{itemize}

\subsection{Variables d'identification}

\begin{itemize}
    \item \textbf{Identifiant} : Identifiant unique de la transaction
    \item \textbf{CodeDecision} : Code de décision associé à la transaction
\end{itemize}

\subsection{Variables temporelles}

\begin{itemize}
    \item \textbf{mois} : Mois de la transaction (2 = février à 11 = novembre)
\end{itemize}

\subsection{Variables de montant}

\begin{itemize}
    \item \textbf{Montant} : Montant de la transaction en euros
    \item \textbf{CA3TR} : Chiffre d'affaires sur 3 mois (transactions)
    \item \textbf{CA3TRetMtt} : Chiffre d'affaires sur 3 mois (montants)
\end{itemize}

\subsection{Variables de scoring}

\begin{itemize}
    \item \textbf{ScoringFP1} : Score de risque FP1
    \item \textbf{ScoringFP2} : Score de risque FP2
    \item \textbf{ScoringFP3} : Score de risque FP3
    \item \textbf{ScoringFP4} : Score de risque FP4
\end{itemize}

\subsection{Variables Verifinance}

\begin{itemize}
    \item \textbf{VerifianceCPT1} à \textbf{VerifianceCPT5} : Indicateurs de vérification comptable
\end{itemize}

\subsection{Autres variables}

\begin{itemize}
    \item \textbf{Enseigne} : Identifiant de l'enseigne commerciale
    \item \textbf{Secteur} : Secteur d'activité
    \item Et autres variables caractérisant la transaction
\end{itemize}

\section{Split temporel}

Conformément aux bonnes pratiques pour les données temporelles, nous avons effectué un split temporel :

\begin{table}[H]
\centering
\caption{Répartition train/test}
\begin{tabular}{lccc}
\toprule
\textbf{Ensemble} & \textbf{Période} & \textbf{Transactions} & \textbf{Taux de fraude} \\
\midrule
Entraînement & Février - Août 2017 & 3 899 362 & 0.60\% \\
Test & Septembre - Novembre 2017 & 747 411 & 0.88\% \\
\bottomrule
\end{tabular}
\end{table}

\textbf{Remarque importante} : Le taux de fraude est plus élevé dans l'ensemble de test (0.88\%) que dans l'ensemble d'entraînement (0.60\%), ce qui suggère une évolution temporelle des patterns de fraude.

\section{Déséquilibre des classes}

Le problème majeur de ce jeu de données est le \textbf{fort déséquilibre des classes} :

\begin{table}[H]
\centering
\caption{Distribution de la variable cible}
\begin{tabular}{lcc}
\toprule
\textbf{Classe} & \textbf{Entraînement} & \textbf{Test} \\
\midrule
Normal (0) & 3 875 940 (99.40\%) & 740 838 (99.12\%) \\
Fraude (1) & 23 422 (0.60\%) & 6 573 (0.88\%) \\
\midrule
\textbf{Ratio} & 1:166 & 1:114 \\
\bottomrule
\end{tabular}
\end{table}

Ce déséquilibre extrême (1 fraude pour 166 transactions normales) nécessite l'utilisation de techniques spécifiques :
\begin{itemize}
    \item Rééchantillonnage (SMOTE, ADASYN, sous-échantillonnage)
    \item Pondération des classes
    \item Métriques adaptées (F1-Score plutôt qu'accuracy)
\end{itemize}

\section{Analyse exploratoire}

\subsection{Distribution des montants}

Les montants des transactions présentent une distribution fortement asymétrique :

\begin{itemize}
    \item Moyenne : 62.31 €
    \item Médiane : 35.00 €
    \item Maximum : plus de 10 000 €
    \item Forte concentration sur les petits montants
\end{itemize}

% \begin{figure}[H]
%     \centering
%     \includegraphics[width=0.8\textwidth]{distribution_montant.png}
%     \caption{Distribution des montants des transactions}
% \end{figure}

\subsection{Variables les plus discriminantes}

L'analyse exploratoire a permis d'identifier les variables les plus discriminantes pour la détection de fraude :

\begin{table}[H]
\centering
\caption{Variables les plus discriminantes (différence de moyenne fraude vs normal)}
\begin{tabular}{lc}
\toprule
\textbf{Variable} & \textbf{Différence (\%)} \\
\midrule
ScoringFP1 & +410\% \\
CA3TR & +290\% \\
ScoringFP2 & -284\% \\
ScoringFP3 & +200\% \\
VerifianceCPT3 & +169\% \\
\bottomrule
\end{tabular}
\end{table}

\subsection{Corrélations}

L'analyse des corrélations a révélé :
\begin{itemize}
    \item Forte corrélation entre les variables VerifianceCPT
    \item Corrélation entre Montant et CA3TRetMtt
    \item Corrélations faibles avec la variable cible (maximum ~0.15)
\end{itemize}

Ces corrélations faibles avec la cible expliquent en partie la difficulté de la tâche de classification.
